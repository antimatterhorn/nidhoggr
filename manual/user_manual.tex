\documentclass[12pt]{article}
\usepackage{geometry}
\usepackage{hyperref}
\usepackage{graphicx}
\usepackage{courier}
\geometry{margin=1in}

\title{Nidhoggr User Manual}
\author{Cody Raskin}
\date{\today}

\begin{document}
	
	\maketitle
	
	\newpage
	
	\begin{center}
		\begin{verbatim}
     _   _ _
 ___|_|_| | |_ ___ ___ ___ ___
|   | | . |   | . | . | . |  _|
|_|_|_|___|_|_|___|_  |_  |_|
v0.8.0            |___|___|
		\end{verbatim}
	\end{center}
	
	\newpage
	
	\tableofcontents
	\newpage
	
	\section{Introduction}
	
The Nidhoggr (pronounced Nith-hewer) is the mythical beast that gnaws at the roots of the world-tree, Yggdrasil. This may or may not have been inspired by my simultaneous loathing and admiration for tree codes.

	\subsection{Purpose of Nidhoggr}

Nidhoggr is a generic physics simulation framework. It is designed to be used as a base for varied physics simulation methods (FVM,FEM,etc) while keeping helper methods like equations of state and integrators generic enough to be portable to a wide variety of methods choices. Nidhoggr's major classes and methods are written in C++ and wrapped in Python using pybind11 to enable them to be imported as Python3+ modules inside a runscript. Python holds and passes the pointers to most objects inside the code, while the integration step is always handled by compiled C++ code. Any Python class that returns the expected data types of the compiled C++ classes can substitute for a precompiled package (e.g. a custom equation of state), though speed will suffer. 

	\subsection{Overview of capabilities}
	
Nidhoggr's capabilities as of \date{\today} are given in Table \ref{tab:component-status}.
\begin{table}[h!]
	\centering
	\caption{Status of major components in Nidhoggr}
	\label{tab:component-status}
	\begin{tabular}{|l|l|l|l|}
		\hline
		\textbf{Component} & \textbf{Working} & \textbf{Development} & \textbf{Planned} \\
		\hline
		Physics & N-body gravity & HLL hydro & SPH\\ 
		& Gravity point sources & FEM &\\
		& Constant direction gravity sources &  &\\
		& Particle kinetics &  &\\
		& Acoustic wave solvers &  &\\
		& Shallow wave equation solvers &  &\\
		& Chemical reaction solvers &  &\\
		\hline
		Equations of State & Ideal gas &  & Helmholtz\\ 
		& Polytrope &  &\\
		\hline
		Time Integrators & Forward Euler &  & Symplectic\\ 
		& 2nd Order Runge-Kutta &  &\\
		& 4th Order Runge-Kutta &  &\\
		\hline
		Meshing & Eulerian grid & FEM & AMR\\
		\hline
		Data IO & Silo &  &\\
		& vtk &  &\\
		& obj &  &\\
		& wav &  &\\
		\hline
		Custom Data Types & Vectors & Elements &\\
		& Tensors &  &\\
		& Cosmologies & & \\
		& Units & & \\
		\hline
		Parallel & OpenMP & & MPI \\
		\hline
	\end{tabular}

\end{table}
	
	\subsection{Intended audience}
	
Nidhoggr's intended audience is computational scientists who want a toy simulation code to scope simple problems with that's easily driveable and scriptable with a Python interface, and anyone who doesn't mind getting their hands dirty writing their own physics packages in a fully abstracted simulation framework. 

\newpage
	
	\section{Installation}
	\begin{itemize}
		\item System requirements
		\item Dependencies
		\item Downloading the source code
		\item Building and installing
	\end{itemize}

\newpage
	
	\section{Getting Started}
	\begin{itemize}
		\item Basic concepts
		\item First run: a simple example
	\end{itemize}

\newpage
	
	\section{Usage}
	\begin{itemize}
		\item Running Nidhoggr
		\item Command-line options
	\end{itemize}

\newpage
	
	\section{Core Concepts}
	
As previously described, Nidhoggr is a compiled C++ codebase that is driven primarily by Python via compiled Python modules. If you're familiar with how NumPy works, you have most of the knowledge you'll need to understand how Nidhoggr works. In fact, while Nidhoggr does have many linear algebra classes and methods builtin, it often works best when paired with other popular Python modules, like NumPy and Matplotlib.

\subsection{Importing Nidhoggr Modules}

The \texttt{nidhoggr.py} file in the tests and examples directories imports all of the compiled Nidhoggr modules for you, and so if you wish to import the entire codebase, you can simply \texttt{from nidhoggr import *}. However, you may choose to import only a subset of the available modules for your specific problem. Consult \texttt{nidhoggr.py} for the full list of modules.

\subsection{Units and Constants}

Nidhoggr is unit agnostic. That is to say, the code does not \textit{prescribe} any particular units for you. However, for most problems, you will want to define your units in order for certain universal constants like $G$ to have their correct values. This is done via the PhysicalConstants object which has two constructor methods. You can either supply the full scope of your desired units with unit length (in meters), unit mass (in kg), unit time (in seconds), unit temperature (in Kelvin), and unit charge (in Coulomb), or just a subset of the first three, wherein temperature will be assumed to be Kelvins and charge will be assumed to be Coulombs. 

For example, in order to simulate something like the Earth with state quantities near $1$, you may choose to instantiate your units like so:

\texttt{PhysicalConstants(6.387e6, 5.97e24, 1.0)}
\\
From these units, Nidhoggr will calculate at the time of the constructor new values for all of the universal constants to use in your chosen physics packages.

Nidhoggr also comes with some helper methods for a handful of frequently used unit systems in \texttt{Units.py}, like \texttt{MKS()}, \texttt{CGS()}, and \texttt{SOL()}. Simply invoke them with \texttt{myUnits = MKS()} if you've imported the \texttt{Units} module.

\subsection{Nodelists and Fields}

	\begin{itemize}
		\item Physics methods
		\item Equations of State
		\item Mesh/grid handling
		\item Boundary conditions
		\item Integrators
		\item The Controller
		\item Periodic work
	\end{itemize}

\newpage
	
	\section{Examples}
	
The \texttt{examples} folder holds Python runscripts that each solve a particular notional physics (or purely calculational) problem. 

\begin{table}[h!]
	\centering
	\caption{Examples included in the main branch.}
	\label{tab:examples}
	\begin{tabular}{|l|l|}
		\hline
		\textbf{File} & \textbf{Purpose}\\
		\hline
		\texttt{cherenkov.py} & Simulates a supersonic (or superluminal) point source \\
		& moving through a medium at a speed greater than c. \\
		\hline		
		\texttt{cosmo.py} & Creates an example cosmology ($\Omega_m$,$\Lambda$,$H_0$) and reports \\
		& the properties of that cosmology at the chosen redshift. \\
		\hline
		\texttt{diffractionGrating.py} & Simulates the transmission of an acoustic wave through \\
		& a diffraction grating.\\
		\hline
		\texttt{imageToStringArt.py} & Creates the instructions for (and previews) an image made \\
		& from strings stretched across a wheel with a chosen number \\
		& of pins. \\
		\hline
		\texttt{oort.py} & Simulates a star passing through the Oort cloud \\
		& and dislodging a comet from its orbit. \\
		\hline
		\texttt{plinko.py} & Simulates the Plinko game. \\
		\hline
		\texttt{relativity.py} & Calculates the time dilation for a relativistic traveler. \\
		\hline
		\texttt{rps.py} & Simulates the destruction of chemical mixtures in a \\
		& rock-paper-scissors-like reaction setup, where \\
		&  $A\to B\to C\to A$.\\
		\hline
		\texttt{tensors.py} & Creates some tensors and does some linear algebra with them.\\
		\hline
		\texttt{vectors.py} & Creates some vectors and does some linear algebra with them.\\
		\hline
		\texttt{waveLogo.py} & Simulates acoustic waves inside a region with Dirichlet\\
		& boundary conditions arranged in a unique fashion.\\
		\hline
	\end{tabular}

\end{table}

	\subsection{Simple test cases}

Many of the Python scripts inside the \texttt{tests} folder stress single components of Nidhoggr, or a small subset of them. For instance, \texttt{waveBox.py} tests the acoustic wave solver with a single oscillatory source in the center of a box with two openings on either end (using Dirichlet boundaries to create the box). 

\newpage
	
	\section{Customization and Extension}
	\begin{itemize}
		\item Modifying source code
		\item Adding new physics modules
		\item Extending the input parser
	\end{itemize}

\newpage
	
	\section{Best Practices}
	\begin{itemize}
		\item Tips for efficient simulation
		\item Debugging guidance
		\item Performance tuning
	\end{itemize}

\newpage
	
	\section{Troubleshooting}
	\begin{itemize}
		\item Common errors and solutions
		\item FAQ
	\end{itemize}

\newpage
	
	\section{Reference}
	\begin{itemize}
		\item Code structure overview
		\item Important classes and functions
		\item File organization
	\end{itemize}

\newpage
	
	\section{Acknowledgments}
	\begin{itemize}
		\item Contributors
		\item Funding and support
	\end{itemize}
	
	\section{License}
	\begin{itemize}
		\item License terms
		\item How to cite Nidhoggr
	\end{itemize}

\newpage
	
	\appendix
	
	\section{Appendix A: Glossary}
	\begin{itemize}
		\item Terms and definitions
	\end{itemize}
	
	\section{Appendix B: Additional Resources}
	\begin{itemize}
		\item Related software
		\item Recommended reading
	\end{itemize}
	
\end{document}
